\documentclass[12pt]{article}
\usepackage[utf8]{inputenc}
\usepackage[russian]{babel}
\usepackage{amsmath, amssymb}
\usepackage{geometry}
\geometry{a4paper, margin=2.5cm}

\begin{document}

\begin{center}
    \LARGE \textbf{Основы машинного обучения}
\end{center}

\vspace{1em}
\noindent\rule{\linewidth}{0.4pt}

\vspace{1em}
\textbf{Основные задачи ML:}
\begin{itemize}
    \item \textbf{Классификация} — определение объектов к определённым классам по общим признакам
    \item \textbf{Регрессия} — прогнозирование величин, функций или событий
    \item \textbf{Ранжирование} — упорядочивание входного набора данных
\end{itemize}

\vspace{1em}
\noindent\rule{\linewidth}{0.4pt}

\section*{Обучающая выборка}

Представление объектов в виде различных векторов данных:

\[
x_i = [x_1, x_2, \ldots, x_n]^T =
\begin{bmatrix}
x_1 \\
x_2 \\
\vdots \\
x_n
\end{bmatrix}
\]

\textbf{Допустим, у нас дана матрица:}

\[
\begin{bmatrix}
x_{11} & x_{12} & \ldots & x_{1n} \\
x_{21} & x_{22} & \ldots & x_{2n} \\
\vdots & \vdots & \ddots & \vdots \\
x_{m1} & x_{m2} & \ldots & x_{mn}
\end{bmatrix}
\]

Здесь \( n \) — количество признаков объекта, а \( m \) — количество самих объектов.

\end{document}
