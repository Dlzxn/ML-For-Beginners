\documentclass[a4paper,12pt]{article} % формат А4, шрифт 12pt

\usepackage[utf8]{inputenc}   % кодировка UTF-8
\usepackage[
    a4paper,
    top=2.5cm,
    bottom=2cm,
    left=3cm,
    right=2cm
]{geometry}
\usepackage{tikz}
\usepackage[most]{tcolorbox}

\usetikzlibrary{positioning}

\usepackage[T2A]{fontenc}     % шрифты для русского
\usepackage[russian]{babel}   % русская локализация
\usepackage{graphicx}
\usepackage{amsmath}
\usepackage{amssymb}
\usepackage{booktabs}   % красивые линии в таблицах
\usepackage{float}      % для [H]
\usepackage{listings}
\usepackage{xcolor} % Для цвета (опционально)
\usepackage{minted}

\lstset{
  language=Python,
  basicstyle=\ttfamily\footnotesize,
  keywordstyle=\color{blue},
  commentstyle=\color{gray},
  stringstyle=\color{orange},
  showstringspaces=false,
  breaklines=true
}

\begin{document}

% Титульный лист
\begin{titlepage}
    \centering
    \vspace{2cm}

    {\Huge \textbf{Основы Машинного Обучения} \par}
    \vspace{1cm}


    \begin{figure}[htbp]
        \centering
        \includegraphics[width=0.9\textwidth]{images/preview.png}
        \label{fig:example}
    \end{figure}
    \vspace{5cm}

    \begin{flushright}
    \textbf{Автор:} Марьичев Алексей \\
    \vspace{3cm}
    \centering
    {\large Нижегородский государственный университет им. Лобачевского\par}
    \end{flushright}

\end{titlepage}

\newpage
\centering
{\Huge \textbf{Предисловие}}

\newpage
\centering
{\Huge \textbf{Словарь}} \\
\begin{itemize}

\vspace{1.3cm}
\item Вектор: \(X = [x_1, x_2, ..., x_n]\)
\item Транспонированый вектор(обозначается \(\X^T): X^T = \begin{bmatrix}
                                                    x_1 \\
                                                     x_2 \\
                                                     ... \\
                                                     x_n \\
\end{bmatrix} \)
    \item

\end{itemize}

\newpage
    \centering
    {\Huge \textbf{Содержание:}}


\newpage
%! Author = alex
%! Date = 04.07.2025

\section*{\Huge \textbf{Глава 1. Математический анализ}}

\newpage
%! Author = alex
%! Date = 04.07.2025

\section*{\Huge \textbf{Глава 2. Линейная Алгебра}}

\newpage
%! Author = alex
%! Date = 04.07.2025

\section*{\Huge \textbf{Глава 3. Теория Вероятности}}

\newpage
%! Author = alex
%! Date = 04.07.2025

\centering
\vspace{1cm}
\section*{\Huge \textbf{Глава 4. Машинное обучение}}
\vspace{0.5cm}

\vspace{1em}
\noindent\rule{\linewidth}{0.4pt}

\vspace{1em}
\textbf{Основные задачи ML:}
\begin{itemize}
    \item \textbf{Классификация} — определение объектов к определённым классам по общим признакам
    \item \textbf{Регрессия} — прогнозирование величин, функций или событий
    \item \textbf{Ранжирование} — упорядочивание входного набора данных
\end{itemize}

\vspace{1em}
\noindent\rule{\linewidth}{0.4pt}
\section*{Раздел 1. Введение в Машинное обучение.}
%------------------------------------------------------------------------------------1.1
\section*{\S 1.1 Обучающая выборка}

Представление объектов в виде различных векторов данных:

\[
    X = [x_1, x_2, \ldots, x_n]^T =  \begin{bmatrix}
x_1 \\
x_2 \\
\vdots \\
x_n
\end{bmatrix}
\]
X - вектор входных данных

\textbf{Допустим, у нас дана матрица:}

\[A =
\begin{bmatrix}
x_{11} & x_{12} & \ldots & x_{1n} \\
x_{21} & x_{22} & \ldots & x_{2n} \\
\vdots & \vdots & \ddots & \vdots \\
x_{m1} & x_{m2} & \ldots & x_{mn}
\end{bmatrix}
\]
\raggedright
Здесь A - матрица входных данных, \( n \) — количество признаков объекта, а \( m \) — количество самих объектов.

Таким же видом представлены и выходные данные:

\[
    Y = [y_1, y_2, \ldots , y_m ]^T = \begin{bmatrix}
                                       y_1 \\
                                       y_2 \\
                                       \vdots \\
                                       y_m
                                       \end{bmatrix}
\]
Y - вектор выходных данных

\newpage
Теперь мы рассмотрим важный вопрос: как же такие объекты как изображения, звук и т. д. могут представляться в виде векторов? \\
\vspace{1cm}
Допустим, на вход задаче подается Изображение:
\begin{figure}[htbp]
    \centering
    \includegraphics[width=0.4\textwidth]{images/snake.png}
    \caption{Пример изображения}
    \label{fig:example_snake}
\end{figure}
\\
Теперь важное замечание — \textbf{размерность вектора n будет зависеть от количества пикселей в изображении} \\
\vspace{1cm}
Например, если изображение 1024 на 256, то размерность вектора будет \(1024 \times 256 = 262144\)
\vspace{0.9cm}

\[
    X = [x_1, x_2, \ldots , x_{262144} ]
\]

\centering
\vspace{1cm}
Теперь \textbf{объединим} эти понятия:
\[
    X' = \{(x_i, y_i) \mid 0 < i \le m \} \text{ — \textbf{размеченные данные (обучающая выборка)}}
\]
Это и является одним из важнейших понятий в области машинного обучения, с которым вы будете встречаться всюду.
\noindent\rule{\linewidth}{0.4pt}

%------------------------------------------------------------------------------------1.2
\newpage
\centering
\section*{\S 1.2 Постановка задачи для модели}
\vspace{1em}
\raggedright
А теперь разберемся с тем, как же модель будет "обучаться": \\
\vspace{0.8cm}
Допустим, у нас есть размеченные данные \((x_i, y_i)\), которые подаются в некоторую модель

\usetikzlibrary{shapes.geometric, arrows.meta}

\tikzset{
    block/.style = {rectangle, draw, fill=blue!20, text centered, minimum height=3em, minimum width=6em},
    arrow/.style = {thick, -{Latex[length=3mm]}},
    data/.style = {circle, draw, fill=green!15, minimum size=2em, inner sep=1pt, text centered}
}

\begin{center}
\begin{tikzpicture}[node distance=2cm and 2.5cm]

% Входные данные
\node[data] (x) {\(x_i\)};
\node[data, below=of x] (y) {\(y_i\)};

% Блок модели
\node[block, right=3cm of $(x)!0.5!(y)$] (model) {Модель};

% Выход
\node[data, right=4cm of model] (output) {Предсказание};

% Стрелки
\draw[arrow] (x) -- ([yshift=+0.8em]model.west);
\draw[arrow] (y) -- ([yshift=-0.8em]model.west);
\draw[arrow] (model) -- (output);

\end{tikzpicture}
\end{center}

В результате из исходных данных мы получили некое предсказание, которое на первых этапах обучения может не иметь ничего общего с правильным ответом. \\
\vspace{1cm}
Теперь представим нашу модель как линейную функцию:
\[
    y(x) = \phi (x, \Delta) \quad (1)
\]
Здесь \(\Delta\) — \textbf{постоянно меняющийся параметр} \\
Его мы будем подстраивать для наиболее точного ответа нашей модели \\
\vspace{1cm}
Для лучшего понимания перейдем к задаче линейной регрессии.
Задана функция:
\[
    y(x, k, b) = kx + b + \psi
\]
Здесь \(k\) и \(b\) — параметры, от которых зависит угол поворота прямой, а так же ее сдвиг.
Т.е. получается, что эта прямая может проходить как угодно, но за счет размеченных данных мы задаем модели желаемый результат:

\begin{figure}[htbp]
    \centering
    \includegraphics[width=0.4\textwidth]{images/lin_regress.png}
    \caption{Линейная регрессия}
    \label{fig:example_lin-regress}
\end{figure}

И получается, что во время обучения модель дает прогнозы все точнее и точнее к желаемому результату. \\
\vspace{1cm}
Но как же наш алгоритм понимает, что ответ надо корректировать?\\
Сейчас мы подошли к еще одному очень важному определению в области ML: \\
\vspace{0.8cm}
\textbf{Функция потерь — функция, которая характеризует потери при неправильном предсказании модели} \\
Примеры таких функций:
\[
    L(x, a) = |a(x) - y(x)| \quad \text{— абсолютная ошибка}
\]
\[
    L(x, a) = (a(x) - y(x))^2 \quad \text{— квадратичная ошибка}
\]

\vspace{0.5cm}

Также введем связное понятие:
\begin{tcolorbox}[colback=gray!10, colframe=black, title=Средний эмпирический риск]
\[
    Q(a) = \frac{1}{l} \sum_{i=1}^l L(a(x_i), y_i)
\]
Среднее значение функции потерь на обучающей выборке.
\end{tcolorbox}

Это средне арифметическое по всем потерям в текущем цикле обучения модели. \\
\vspace{0.5cm}
Вспомним формулу (1):
Наша задача — минимизировать средний эмпирический риск за счет изменения параметра \(\Delta\) \\
\vspace{1.5cm}

%------------------------------------------------------------------------------------1.3

\centering
\section*{\S 1.3 Линейная модель}
\vspace{0.8cm}
\raggedright

Рассмотрим функцию \( y = kx + b + \psi \).

В процессе обучения модели на данной функции перед нами
будет стоять задача подобрать такие \( k^{\prime} \) и \( b^{\prime} \),
чтобы сама функция \( y \) наиболее точно отображала желаемый результат.

\vspace{0.5cm}

Но что если мы попробуем выразить функцию через характеристики объекта?

Допустим, у нас есть \( \phi_1 \), \( \phi_2 \):

\[
    y = f_1(x)\phi_1 + f_2(x)\phi_2 + \psi
\]

Здесь \( f_1(x) \) — первая характеристика объекта, а \( f_2(x) \) — вторая.

Очевидно, что если \( f_1(x) = x \), \( f_2(x) = 1 \), а \( \phi_1 = k^{\prime} \), \( \phi_2 = b^{\prime} \),
то формула сводится к изначальной:

\[
    y(x) = k^{\prime}x + b^{\prime}
\]

Итак, линейная модель:
\[
    a(x) = \sum_{i=0}^{n} f_i(x)\phi_i
\]

Рассмотрим конкретный пример

\centering
\section*{\S 1.3 BETA Переобучение}
\vspace{0.8cm}
\raggedright

Нам дана функция:
\[
    f(x) = x^2 + x + 1
\]
\begin{figure}[htbp]
    \centering
    \includegraphics[width=0.4\textwidth]{images/linear_data.png}
    \caption{Данные для функции}
    \label{fig:example_linFunc}
\end{figure}

Сначала может показаться, что мы можем описать данную функцию с
помощью одной характеристики: \(a(x) = f(x) \phi\) \\
\begin{figure}[htbp]
    \centering
    \includegraphics[width=0.4\textwidth]{images/error_data.png}
    \caption{Ошибка модели}
    \label{fig:example_error_of_model}
\end{figure}
Но в таком случае получится прямая линия, лишь по очертаниям похожая на нашу кривую. \\
\vspace{0.7cm}
В таком случае нам поможет полином:
\[
\begin{cases}
    f_0(x) = \text{const} \\
    f_1(x) = x \\
    f_2(x) = x^2 \\
    \vdots \\
    f_n(x) = x^n
\end{cases}
\quad \textbf{-- все характеристики}
\]

Таким образом наша модель (при \(f_0 = 1\)):
\[
a(x) = \sum_{i=0}^n f_i(x) \phi_i = \phi_0 + x\phi_1 + \ldots + x^n \phi_n
\] \\
\begin{tcolorbox}[colback=gray!10, colframe=black, title=Важно]
    Система характеристик является \textbf{Линейно Независимой} \\
\end{tcolorbox}

\textbf{Почему характеристики не могут быть линейно зависимыми?} \\
\vspace{0.5cm}

Допустим задана система:
\[
    \begin{cases}
        f_0(x) = 1 \\
        f_1(x) = x \\
        f_2(x) = x + 5
    \end{cases}
\]
Получается, что \(f_2(x) = f_1(x) + 5\) \\
Следовательно, если одну из характеристик можно выразить через другие, то зачем же она вообще нужна? Получается, что
она просто является лишней в нашей системе и можно справиться без нее.

% ПЕРЕПИСАТЬ ГЛАВУ

\newpage
\centering
\section*{\S 1.4 Степень переобучения модели}
\vspace{0.8cm}
\raggedright

\textbf{1.4.1 Оценка по отложенной выборке (hold-out)} \\
\vspace{0.5cm}
Для данного метода размеченные данные делят на две части: \\
\vspace{0.2cm}
\begin{itemize}
    \item Обучающие
    \item Отложенная (hold-out)
\end{itemize}
\textbf{Обычно данные делят в соотношении 70:30.}

Цель: сравнить качество модели на данных, используемых при обучении, с новыми данными того же характера.
Для этого строят два новых параметра: \textbf{\(Q(a, X)\) — для обучающих данных и \(Q'(a, X')\) — для отложенных данных.} \\
\vspace{0.3cm}

В результате, если средний эмпирический риск для обучающих данных меньше, чем для отложенных,
то модель следует подкорректировать для лучшего показателя. \\
\vspace{0.5cm}

\textbf{1.4.2 Скользящий контроль (leave-one-out)} \\
\vspace{0.5cm}

Допустим, нам даны \(n\) различных размеченных данных: \(X = (x_1, x_2, \ldots, x_n)\) \\
Данный метод основан на том, что мы построим \(n\) таких моделей, что: \\
\vspace{0.3cm}
\[
\begin{cases}
    a_1(x): X_1 = (x_2, x_3, \ldots, x_n) \\
    a_2(x): X_2 = (x_1, x_3, x_4, \ldots, x_n) \\
    \ldots \\
    a_n(x): X_n = (x_1, x_2, \ldots, x_{n-1})
\end{cases}
\]
\vspace{0.2cm}
Т.е. мы построили \(n\) различных моделей \(a_i(x)\), таких, что каждая из них обучалась на наборе
данных размерностью \(n-1\) (для \(a_i\)-ой модели убирали \(x_i\)-ый вектор данных )\\
\vspace{0.3cm}

Ну а конечная модель \(a(x) = F(a_1, a_2, \ldots, a_n)\) \\
\vspace{0.3cm}

На больших наборах данных этот способ требует огромной вычислительной мощи,
потому он почти не используется на практике. \\
\vspace{0.5cm}
\textbf{1.4.3 Кросс-валидация (cross-validation, k-fold)} \\
\vspace{0.5cm}
Очень похожий на скользящий контроль метод, но различие состоит в том, что здесь мы разбиваем входные данные
на некоторые группы и составляем из них модели: \\
\begin{figure}[htbp]
    \centering
    \includegraphics[width=0.3\textwidth]{images/sxema.png}
    \caption{Схема k-fold кросс-валидации}
    \label{fig:example_kross_validator}
\end{figure}
Этот метод позволяет строить модели, которые будут обладать \(\textbf{лучшими обобщающими способностями,
при меньшем количестве вычислений.}\)

\vspace{0.5cm}
\textbf{1.4.4 Обобщение моделей} \\
\vspace{0.5cm}
В прошлых частях мы встречались с обобщением модели: \(a(x) = F(a_1(x), a_2(x), \ldots, a_n(x))\), но не упоминалось,
как же это делается на самом деле.
В основном используют два метода для обобщения моделей, познакомимся с ними поближе: \\
\vspace{0.5cm}

1) \(\textbf{Выбор одной модели с лучшими показателями}\) \\
\vspace{0.3cm}
Допустим у нас есть модели: \\
\[
    \begin{cases}
    a_1(x) \\
    a_2(x) \\
    \ldots \\
    a_n(x)
    \end{cases}
\]
Среди них мы выбираем ту модель, у которой \(\textbf{средний эмпирический риск минимальный}\).
Но на самом деле, даже если этот показатель и наименьший, это не означает, что модель лучше остальных. \\
\vspace{0.3cm}
2) \(\textbf{Выбор наиболее часто встречающегося результата}\) \\
\vspace{0.3cm}
Допустим у нас есть модели, которые выдают определенный результат \(\in (0, 1)\): \\
\[
    \begin{cases}
    a_1(x) = 1 \\
    a_2(x) = 1 \\
    \ldots \\
    a_{n-1}(x) = 1 \\
    a_n(x) = 0
    \end{cases}
\]
Чаще всего встречается ответ 1, а значит мы его примем за верный. \\
Но данный способ \(\textbf{требует большой вычислительной мощи}\),
т.к. предсказания нам будет давать уже не одна, а \(n\) моделей. \\


%------------------------------------------------------------------
\newpage
\centering
\section*{\S 1.5 Уравнение гиперплоскости}
\vspace{0.8cm}
\raggedright
Рассмотрим
\begin{figure}[htbp]
        \centering
        \includegraphics[width=0.45\textwidth]{images/examples_graph_4.png}
        \label{fig:example_4}
\end{figure}
Здесь прямая в двумерном пространстве делит два класса предметов, т.е. по левую часть от прямой располагаются предметы,
относящиеся к одному классе, а по правую-к другому классу. \\
\vspace{0.5cm}

Обратимся к линейному уравнению:
\[
    w_1 x_1 + w_2 x_2 + w_0 = 0
\]
Вектор \(w = (w_1, w_2)^T\) является нормалью к гиперплоскости, т.е. ортогонален всем векторам, лежащим в этой плоскости.


\begin{figure}[htbp]
        \centering
        \includegraphics[width=0.45\textwidth]{images/examples_graph_41.png}
        \label{fig:example_41}
\end{figure}
\vspace{0.5cm}

\textbf{Докажем это:} \\

Пусть \(x\) — произвольный вектор, лежащий в гиперплоскости. Тогда:

\[
w \cdot x = \|w\| \|x\| \cos(\theta)
\]

Поскольку \(x\) лежит в гиперплоскости, а \(w\) — нормаль, угол между ними \(90^\circ\), следовательно:

\[
\cos(\theta) = 0 \Rightarrow w \cdot x = 0
\]

\(\blacksquare\)

\vspace{0,5cm}
Но как же  нам теперь отличать объекты одного класса от объектов другого класса с использованием полученных знаний?

Все очень просто, рассмотрим углы между ортогональной к плоскости прямой и прямой до объекта.
\begin{figure}[htbp]
        \centering
        \includegraphics[width=0.45\textwidth]{images/examples_graph_42.png}
        \label{fig:example_42}
\end{figure} \\

Заметим, что угол a является \(\textbf{острым углом}\), как и любой другой угол между нормальную к гиперплоскости
и прямой до объекта такого же класса.(cos(a) > 0)\\
\vspace{0.3cm}

а вот угол b является является \(\textbf{тупым}\), как и любой другой угол между нормальную к гиперплоскости
и прямой до объекта такого же класса.(cos(a) < 0) \\
\vspace{0.3cm}

А значит скалярное произведение (w, x) для одного класса будет положительным,
а для другого отрицательным(следует из знака cos) \\
Более того, мы можем описать такую функцию как:
\[
    a(x, w) = sign((w, x)) = \begin{cases}
                                -1, \ (w, x) < 0 \\
                                \ 1,  \ \ \ (w, x) > 0
                                \end{cases}
\]
При нуле объект не будет определен ни к одному классу.


%------------------------------------------------------------------------


\newpage
\section*{\S 1.6 Задача бинарной классификации}
\vspace{0.8cm}
Теперь перейдем к задаче на практике, нам заданы характеристики объектов:
\vspace{0.5cm}

\begin{table}[H]
    \centering
    \begin{tabular}{rrrr}
        \toprule
        \textbf{№} & \textbf{Ширина} & \textbf{Длина} & \textbf{Жук} \\
        \midrule
        1  & 10 & 50 & гусеница \\
        2  & 20 & 30 & божья коровка \\
        3  & 25 & 30 & божья коровка \\
        4  & 20 & 60 & гусеница \\
        5  & 15 & 70 & гусеница \\
        6  & 40 & 40 & божья коровка \\
        7  & 30 & 45 & божья коровка \\
        8  & 20 & 45 & гусеница \\
        9  & 40 & 30 & божья коровка \\
        10 & 7  & 35 & гусеница \\
        \bottomrule
    \end{tabular}
    \caption{Характеристики насекомых по ширине и длине}
    \label{tab:bugs}
\end{table}

Здесь обозначим -1 как гусеницу, 1 как божью коровку.
\noindent\rule{\linewidth}{0.4pt}

Критерий качества для разделяющей линии сформулировал Фрэнк Розенблатт,
он определил это как количество неверных классификаций:
\[
    Q(a, X) = \sum_{i=1}^n [a(x_i) \ne y_i}]
\] \\
Квадратные скобки - индикатор ошибки, они переводят True/False в 1/0 соответственно(нотация Айверсона):
\[[a(x_i) \ne y_i] \in \{0, 1\}\]

В данной задаче мы можем посчитать критерий качества иначе, в виду того, что \(y \in \{-1, 1\}\)
Мы можем утверждать, что \(y_i a(x_i)\)) будет положительный при верной классификации и отрицательным при неверной.
\[Q(a, X) = \sum_{i = 1}^n [y_i a(x)] < 0\]
Такое произведение очень часто используется в задачах бинарной классификации и обозначают как
\(M = y_i a(x_i)\)
Его называют \(\textbf{Отступом}\). Эта величина может показывать не только признак верной классификации,
но и насколько далеко отстоит образ от разделяющей плоскости. \\
\vspace{0.4cm}

Теперь приступим к написанию такого алгоритма.
У нас задано множество признаков, а так же правильная классификация, наша задача - это создать код,
который от произвольного начального значения коэффициентов будет изменять эти значения в нужную сторону,
пока не найдет тот, что верно подберет разделяющую прямую для нашей задачи(т.е. по критерию качества сумма будет равняться 0). \\
В данном случае у нас два признака, а потому вектор \(W = [w_1, w_2]\) будет двумерным. Зафиксируем \(w_2\) = -1
для простоты примера.\\

\vspace*{0.3cm}
\(\textbf{Реализация на Python:}\)
\vspace*{0.3cm}
\begin{minted}{python}
import numpy as np
import matplotlib.pyplot as plt

x_train = np.array([[10, 50], [20, 30], [25, 30], [20, 60],
[15, 70], [40, 40], [30, 45], [20, 45], [40, 30], [7, 35]])
y_train = np.array([-1, 1, 1, -1, -1, 1, 1, -1, 1, -1])

n_train = len(x_train)
w = [0, -1]
a = lambda x: np.sign(x[0]*w[0] + x[1]*w[1])
N = 50
L = 0.1
e = 0.1                                         # небольшая добавка для w0

last_error_index = -1

for n in range(N):
    for i in range(n_train):                # перебор по наблюдениям
        if y_train[i]*a(x_train[i]) < 0:
            w[0] = w[0] + L * y_train[i]
            last_error_index = i

    Q = sum([1 for i in range(n_train) if y_train[i]*a(x_train[i]) < 0])
    if Q == 0:      # показатель качества классификации (число ошибок)
        break       # останов, если все верно классифицируем

if last_error_index > -1:
    w[0] = w[0] + e * y_train[last_error_index]

print(w)

line_x = list(range(max(x_train[:, 0])))
line_y = [w[0]*x for x in line_x]

x_0 = x_train[y_train == 1]                 # формирование точек для 1-го
x_1 = x_train[y_train == -1]                # и 2-го классов

plt.scatter(x_0[:, 0], x_0[:, 1], color='red')
plt.scatter(x_1[:, 0], x_1[:, 1], color='blue')
plt.plot(line_x, line_y, color='green')

plt.xlim([0, 45])
plt.ylim([0, 75])
plt.ylabel("длина")
plt.xlabel("ширина")
plt.grid(True)
plt.show()

\end{minted}
\noindent\rule{\linewidth}{0.4pt}

И после запуска получаем такой график: \\

\begin{figure}[htbp]
        \centering
        \includegraphics[width=0.45\textwidth]{images/examples_graph_6.png}
        \label{fig:example_6}
\end{figure} \\

Отлично! Мы провели классификацию двух объектов!






\section*{Раздел 2. Алгоритмы Градиентного спуска.}
1

\newpage
\section*{\Huge \centering \textbf{Решение задач}} \\
\vspace{1.1cm}

1) На графике представлена разделяющая линия в пространстве двух признаков

\begin{figure}[htbp]
        \centering
        \includegraphics[width=0.5\textwidth]{images/examples_graph_1.png}
        \label{fig:example}
\end{figure}
Требуется найти вектор коэффициентов \([w_0, w_1, w_2]^T\), удовлетворяющий линейной системе: \\
\[w_1 x_1 + w_2 x_2 + w_0 = 0\] \\
\vspace{1cm}

2) На графике представлена разделяющая линия в пространстве двух признаков

\begin{figure}[htbp]
        \centering
        \includegraphics[width=0.5\textwidth]{images/examples_graph_2.png}
        \label{fig:example}
\end{figure}
Требуется найти вектор коэффициентов \([w_0, w_1, w_2]^T\), удовлетворяющий линейной системе: \\
\[w_1 x_1 + w_2 x_2 + w_0 = 0\]
\vspace{1cm}

3) На графике представлена разделяющая линия в пространстве двух признаков

\begin{figure}[htbp]
        \centering
        \includegraphics[width=0.5\textwidth]{images/examples_graph_3.png}
        \label{fig:example}
\end{figure}
Требуется найти вектор коэффициентов \([w_0, w_1, w_2]^T\), удовлетворяющий линейной системе: \\
\[w_1 x_1 + w_2 x_2 + w_0 = 0\]
\vspace{1cm}











\end{document}